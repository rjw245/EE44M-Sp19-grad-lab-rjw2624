
\begin{abstract}
    In a multi-tasking embedded operating system, it's important to guarantee complete isolation of individual tasks. This is provided in part by the ``virtualization'' of the CPU: tasks are written as monolithic programs which depend on the OS to fairly share compute resources among all tasks in the system. In effect, a knowledge barrier is put in place between tasks that helps simplify the process of designing complex systems. In an ideal world, these tasks could not interact with one another except through explicit synchronization channels; in reality, however, tasks are able to tamper with each other's memory, since all share the same address space. Additional hardware is needed to strengthen isolation guarantees and prevent tasks from reading and writing one another's private data. The Memory Protection Unit (MPU) is a peripheral common to most modern processors including the TM4C123G. In this report, I will describe how I have leveraged the TM4C123G's MPU to achieve protection of task and OS memory while minimizing additional context switch overhead. Loaded processes' code, data, stack(s) and heap are protected from other tasks, and OS memory is protected from loaded processes. Tasks compiled with the OS are able to access OS code and global data in order to execute, but are still isolated from one another. My implementation can mutually isolate up to twenty-nine foreground tasks with a 16KB heap.
\end{abstract}

