
\begin{abstract}
    In a multi-tasking operating system, it's important to guarantee complete isolation of individual tasks. This is provided in part by the ``virtualization'' of the CPU: tasks are written as monolithic programs which depend on the OS to fairly share compute resources among all tasks in the system. In effect, a knowledge barrier is put in place between tasks that helps simplify the process of designing complex systems. In an ideal world, these tasks could not interact with one another except through explicit synchronization channels; in reality, however, tasks are able to tamper with each other's memory, since all share the same address space. Additional hardware is needed to strengthen isolation guarantees and prevent tasks from reading and writing one another's private data. The Memory Protection Unit (MPU) is a peripheral common to most modern processors including the TM4C123G. In this report, I will describe how I have leveraged the TM4C123G's MPU to achieve protection of a given task's stack and any memory allocated to it in the heap. My implementation can mutually isolate up to sixteen foreground tasks.
\end{abstract}
