
\chapter{Evaluation}

\section{Verifying Stack Protection}
% Create an example program. Show that accessing another task's stack memory yields fault.

\section{Verifying Heap Protection}
% Create an example program. Show that accessing another task's heap memory yields fault.

\section{Demonstrating Heap Flexibility}
% Characterize fragmentation when sixteen tasks requests varying amounts of heap memory.
% Show that my solution better handles dynamic memory use.

\section{Measuring Context Switch Overhead}

\begin{table}[h]
\begin{tabular}{|l|l|}
\hline
\textbf{Config}                                                                        & \textbf{Context Switch Time} \\ \hline
No Memory Protection                                                                   & 2.833 microseconds           \\ \hline
\begin{tabular}[c]{@{}l@{}}Memory Protection with\\ Heavy Reconfiguration\end{tabular} & 9.333 microseconds           \\ \hline
\begin{tabular}[c]{@{}l@{}}Memory Protection with\\ Lite Reconfiguration\end{tabular}  & 7.292 microseconds           \\ \hline
\end{tabular}
\label{tab:ctxsw}
\end{table}

In Table \ref{tab:ctxsw}, I measure the context switch duration for no memory protection, memory protection with heavy reconfiguration, and my design: memory protection with lite reconfiguration. The implementation with heavy reconfiguration fully initializes each of the four MPU regions of the heap during each context switch. This approximates a solution in which each task and/or process in the system maintains its own set of MPU regions which must be programmed into the MPU prior to running. This is much ``heavier'' reconfiguration than merely masking subregions of a constant region configuration as in my solution. The addition of memory protection does lengthen the context switch duration. That said, the context switch is shorter than it would otherwise be thanks to the use of subregions, which enable relatively light reconfiguration of the MPU during a context switch compared to heavier approaches. This works out to a reduction in context switch duration of 21.9\%.
